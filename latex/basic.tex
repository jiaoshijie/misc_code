%!TEX program = xelatex
% latex使用 % 进行注释

% 每一个latex命令使用反斜线 (\) 开始
\documentclass[fontset=adobe, 12pt]{article}

\usepackage{ctex}     % 使用中文字
\usepackage{caption}  % 展示数据的包
\usepackage{float}    % 展示数据的包
\usepackage{hyperref} % 使用超链接的包
\usepackage{graphicx} % 使用图片的宏包

\author{焦世杰}
\date{\today}
\title{Learn \LaTeX \hspace{1pt} in Y Minutes!}


\begin{document}
% 通过使用这个命令来生成首页
\maketitle
\newpage  % 新的一页
\tableofcontents  % 创建目录

\newpage

\begin{abstract}
  \LaTeX \hspace{1pt} documentation written as \LaTeX! How novel and totally not my idea!

\end{abstract}

% 章节指令
\section{Introduction}
Hello, 我的名字是焦世杰 and together we're going to explore \LaTeX!

\section{Another section}
This is the text for another section. I think it needs a subsection.

\subsection{This is a subsection}
I think we need another one

\subsection{Pythagoras}
Much better now.
\label{subsec:pythagoras}

% 使用*可以取消LaTex内置的编号功能
% 这在其他指令中也有效
\section*{This is an unnumbered section}
因为并不是所有的章节都要被标号

\section{Some Text notes}
% \section{Spacing}
\LaTeX \hspace{1pt} is generally pretty good about placing text where it should go. If a line needs to break you add \textbackslash\textbackslash \hspace{1pt} to source code.
\section{Lists}
Lists are one of the easiest things to create in \LaTeX! I need to go shopping tomorrow, so let's make a grocery list.
\begin{enumerate}
  % \item 使枚举增加一个单位
  \item Salad.
  \item 27 watermelon
  \item A single jackrabbit
  \item[how many?] Medium sized squirt guns.

  Not a list item, but still part of the enumerate.
\end{enumerate}

\section{Math}
使用 \LaTeX \hspace{1pt} 的一个最主要的方面是学术论文和技术文章
通常在数学和科学的领域
因此我们需要在文章中插入特殊符号!

数学符号极多, 远超出你能在键盘上找到的那些;\\
集合关系符, 箭头, 操作符, 希腊字符等等

集合与关系在数学文章中很重要\\
如声明所有 x 属于 X $\forall$ x $\in$ X.
% 我们平时的编写是在text-mode, 然而数学符号只是在 math-mode 中存在
% text-mode 进入 math-mode 使用 $ 操作符
% 也可以使用 \[\] 来进入 math-mode

\[a^2 + b^2 = c^2\]

My favorite Greek letter is $\xi$. I also like $\beta$, $\gamma$ and $\sigma$.
I haven't found a Greek letter yet that \LaTeX \hspace{1pt} doesn't know about!

常用的函数操作符同样很重要:\\
trigonometric function ($\sin$, $\cos$, $\tan$),\\
logarithms 和 exponentials ($\log$, $\exp$),\\
limits ($\lim$), etc.

在LaTex 指令中预定义\\
让我们写一个等式看看发生了什么:\\
$\cos(2\theta) = \cos^{2}(\theta) - \sin^{2}(\theta)$

分数可以写成一下形式:
% 10 / 7
$$ ^{10}/_{7}$$

% 相对比较复杂的分数可以写成
% \frac{numerator}{denominator}
$$ \frac{n!}{k!(n-k)!}$$

我们同样可以插入公式(equations) 在环境``equations environment''下。
\begin{equation}
  c^2 = a^2 + b^2
  \label{eq:pythagoras}
\end{equation}

引用我们的新等式!\\
Eqn.~\ref{eq:pythagoras} is also known as the Pythagoras Theorem which is also the subject of Sec.~\ref{subsec:pythagoras}. A lot of things can be labeled: figures, equations, sections, tec.

求和(Summations)与整合(Integrals)写作sum 和 int:

\begin{equation}
  \sum_{i=0}^{5} f_{i}
\end{equation}
\begin{equation}
  \int_{0}^{\infty} \mathrm{e}^{-x} \mathrm{d}x
\end{equation}

\newpage

\section{Figures}

让我们插入图片. 图片的放置非常微妙.\\
我在每次使用时都会查找可用选项.

\begin{figure}[H] % H 是放置选项的符号
  \centering % 图片在本页居中
  % 宽度缩放为页面的0.8倍
  \includegraphics[width=0.8\linewidth]{picture.jpg}
  % 需要使用想象力决定是否语句超出编译预期
  \caption{Right triangle with sides $a$, $b$, $c$}
  \label{fig:right-triangle}
\end{figure}

\subsection{Table}
插入表格与插入图片方式相同

\begin{table}[H]
  \caption{Caption for the Table.}
  \centering
  % 下方的 {} 描述了表格中没一行的绘制方式
  % 同样, 我在每次使用时都会查找可用选项
  \begin{tabular}{c|cc}
    Number & Last Name & First Name \\ % 每一列被&分开
    \hline
    1 & Biggus & Dickus \\
    2 & Monty & Python
  \end{tabular}
\end{table}

\section{Getting \LaTeX \hspace{1pt} to not compile something (i.e. Source Code)}

现在增加一些源代码在 \LaTeX \hspace{1pt} 文档中,
我们之后需要 \LaTeX \hspace{1pt} 不翻译这些内容而仅仅是把他们打印出来.
这里使用 verbatim environment

% 也有其他库存在(如: minty, lstlisting, 等)
% 但是 verbatim 是最基础和简单的一个
\begin{verbatim}
  print("Hello World!")
  a%b; % 在这一环境中可以使用 %
  random = 4; # decided by fair random dice roll
\end{verbatim}

% \newpage

\section{Compiling}

现在你大概想了解如何编译这个美妙的文档\\
然后得到饱受称赞的 \LaTeX \hspace{1pt} pdf文档\\
使用 \LaTeX \hspace{1pt} 组合步骤:
\begin{enumerate}
  \item Write the document in plain text (the ``source code'').
  \item Compile source code to produce a pdf.

  The compilation step looks like this (in Linux):

  \begin{verbatim}
    > pdflatex main.tex
    or
    > xelatex main.tex
  \end{verbatim}
\end{enumerate}

许多 \LaTeX \hspace{1pt}编译器把步骤1和2放在同一个软件中进行了整合, 所以你可以只看步骤1完全不用看步骤2

步骤2同样在以下情景中使用 \footnote{以防万一, 当你使用引用时(如 Eqn.~\ref{eq:pythagoras}), 你将需要多次运行步骤2来生成一个媒介文件 *.aux}.
% 同时这也是在文档中增加脚标的方式

在步骤1中, 用普通文本写入格式化信息, 步骤2的编译阶段则注意步骤1中定义的格式信息.

% \newpage

\section{Hyperlinks}
同样可以在文档中加入超链接

使用如下命令在序言中引入库:
\begin{verbatim}
  \usepackage{hyperref}
\end{verbatim}

有两种主要的超链接方式\\
\url{https://learnxinyminutes.com/docs/zh-cn/latex-cn/}, 或
\href{https://learnxinyminutes.com/docs/zh-cn/latex-cn/}{shadowed by text}
% 不可以增加特殊空格和符号, 因为这将会造成编译错误

这个库同样在输出PDF文档时制造略缩的列表, 或在目录中激活链接

\section{End}

这就是全部内容了!

% 通常, 你会希望文章中有个引用部分(参考文献)
% 最简单的建立方式是使用书目提要章节
\begin{thebibliography}{1}
  % 与其他列表相同, \bibitem 命令被用来列出条目
  % 每个记录可以直接被文章主体引用
  \bibitem{latexwiki} The amazing \LaTeX \hspace{1pt} wikibook: {\em https://en.wikibooks.org/wiki/LaTeX}
  \bibitem{latextutorial} An actual tutorial: {\em http://www.latex-tutotial.com}
\end{thebibliography}

% 结束文档
\end{document}
